\section{Inledning}
\label{sec:inledning}
I denna laboration skall ni i vattenlösning studera hastigheten för
(framåt)reaktionen mellan järn(III) joner och thiocyanat joner.

\begin{center}
\begin{tabular}{ccc}
  \ce{Fe^3+ + SCN- <=>>[k_f][k_b] FeSCN^2+} & %
    $\log_{10} \left( \beta_1 %
     % (I=\SIrange{1.0}{1.2}{\molar}) %
    / \si{\per\Molar} \right) = \num{2.11(4)}$ %
  % & $\Delta H_1 = \SI{-6.7}{\kilo\joule\per\mole}$
\end{tabular}
\end{center}
komplexbildningskonstanten för thiocyanatojärn(II) är från
Referens\cite{bahta_critical_1997}.

\ce{FeSCN^2+} är starkt rödfärgat ($\lambda_{max}=\SI{450}{\nm}$) medan
reaktanterna är färglösa. Det finns komplex med fler än en thiocyanat
jon, men så länge koncentrationen av \ce{SCN-} är tillräckligt låg kan vi
förbise dessa.

\subsection{Mål}
Målet med laborationen är att ni självständigt skall undersöka både
temperatur- och jonstyrkeberoendet av hastigheten för
komplexbildningsreaktionen. Hur ni väljer att analysera dessa aspekter är
upp till er själva, men förslagsvis utgår ni från någon väl etablerad
teori som introducerats i denna kurs. Utöver val av teoretisk behandling
kommer ni själva få välja experimentella förhållanden vid era försök.

Slutligen skall ni redovisa era resultat och slutsatser i en
laborationsrapport som följer strukturen för en vetenskaplig artikel.
