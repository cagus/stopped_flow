\section{Experimentellt genomförande}
\label{sec:exper}
För att kunna analysera reaktionshastigheten behöver ni kunna följa
reaktionens gång som funktion av tid. Till ert förfogande kommer ni ha en
s.k. ``stopped-flow''-utrustning (se \cref{fig:stopped-flow}).

\begin{figure}[center]
  \centering
  \includegraphics[scale=0.2]{fig/stopped_flow.png}
  \caption{Schematisk representation av stopped-flow utrustning}
  \label{fig:stopped-flow}
\end{figure}

Blandkammaren är termostaterad med hjälp av vattenbad som ni själva får
välja temperatur för. Ni kan börja era försök vid
rumstemperatur, temperaturintervallet ni kan arbeta inom styrs av
temperaturen av kallvattnet i husets ledningar (säg \SI{12}{\celsius})
och vattenbadets övre gräns (säg \SI{50}{\celsius}). 

Kyvettens längd är \SI{1}{\centi\metre} (vilket
tillsammans med extinktionskoefficienten  är något ni har nytta av att
veta när ni väljer koncentrationsintervall för era reaktantlösningar).

Ni kommer att använda er utav en dator med ett interface
skrivet i LabView. Handhavandet av programmet är beskrivet i
\cref{sec:handhavande}. Från varje försök kommer ni att erhålla dataserier med
absorbans som funktion av tid vid en våglängd som ni själva väljer. Dessa
tidsserier kommer sparas i form av textfiler som ni sparar på USB minne
som ni själva tar med er till labb.

\subsection{Stamlösningar}
För att bereda era två reaktantlösningar kommer ni ha tillgång till följande
stamlösningar:

\begin{itemize}
\item \SI{2}{\Molar} \ce{NaClO4}
\item \SI{2}{\Molar} \ce{HClO4}
\item \SI{100}{\milli\Molar} \ce{KSCN}
\item \SI{100}{\milli\Molar} \ce{Fe(NO3)3} + \SI{200}{\milli\Molar} \ce{HClO4}
\end{itemize}

Eftersom blandkammaren inte är perfekt kan det vara bra att försäkra sig
om att båda reaktantlösningarna får samma jonstyrka och pH.

\subsection{Handhavande}
\label{sec:handhavande}
Nedan finner ni en instruktion för handhavandet av
laborationsutrustningen och programvaran. Enheten för tidsangivelserna är
\si{\milli\second} i programmet.

\subsubsection{Förberedelser}
\begin{itemize}
\item Tänd spektrofotometerns lampan minst 5 minuter innan mätningarna.
\item Kontrollera att vattennivån är inom de markerade gränserna för vattenbadet.
Starta termostat och lägg i extern termometer.
\item Kontrollera att kylvattnet är på med hjälp av den röda flödesmätaren.
\item Sätt på kranvattnet.
\item Starta programmet ``StopFlowSpect'' på datorn.
\item Ställ in och invänta rätt temperatur OBS! Vid alla mätningar och
  kalibreringar ska knappen för ``Start/Stop Acquisition`` vara röd. Om
  programmet stängs av måste kalibreringen (med dest. vatten) göras om,
  tryck därför aldrig på ``Exit''. Undvik att luftbubblor kommer in i
  kyvetten.
\end{itemize}
\subsubsection{Kalibrering}
\begin{itemize}
\item Ställ ``Integration time'' till 2 ms.
\item Blockera strålgången med metallplåten. Tryck på ``Dark''. Ta bort
  metallplåten.
\item Injicera destillerat vatten. Tryck på ``Ref''.
\item Kontrollera att transmittansen och absorptionen ser ut som de
  bör: en blå kurva motsvarande spektrumet för lampan och en röd för
  bakgrundsnivå i mörker.
\item Byt sprutor och injicera tillräckligt med reaktantlösning så att
  komplexet bildas i kyvetten.
\item Gå in i absorptionstfönstret och bestäm vid vilken våglängd
  mätningarna ska göras genom att välja ``Peak'' i Integral/peak-menyn
  och ``drag-drop'':a den vertikala linjen där ni vill mäta. Beakta att det
  viktiga måttet vid val av våglängd är förhållandet mellan signal och brus
  (``signal-to-noise-ratio''), denna få ni själva bedöma var den är som
  bäst genom att aktivera ``Start/stop acquisition''.
\end{itemize}
\subsubsection{Mätning}
\begin{itemize}
\item Gå in på fliken ``Time mode''.
\item Välj ``As fast as possible'', ``Save to file'' och ``Use a trigger''.
\item Tryck på ``Start'' så att den gröna lampan lyser (triggern aktiveras).
\item Vinkla samtliga T-kranar nedåt.
\item Injicera snabbt in ny reaktantlösning. Håll kvar greppet i 4 sekunder.
\item Välj alltid ``Replace'' när frågerutan dyker upp (se till att
  kopiera och döpa om {\tt data.txt} ifall ni är nöjda med en körning, ni
  hittar en genväg till filen på skrivbordet).
\end{itemize}

%%% Local Variables:
%%% mode: latex
%%% TeX-master: "../main"
%%% ispell-local-dictionary: "swedish"
%%% End:
