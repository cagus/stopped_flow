\section{Laborationsrapport}
\label{sec:rapport}
Laborationsrapporten skall skickas till laboratorieassistenten en
studievecka efter laborationens genomförande.

\subsection{Disposition}
\begin{description}
  \item[Sammanfattning] \hfill \\
    Vad som på engelska är känt som ``abstract''. Här sammanfattar ni i
    ett stycke vad som gjordes och anger de viktigaste resultaten.
  \item[Inledning] \hfill \\ 
    Bakgrund till laborationen.
  \item[Teori] \hfill \\ 
    Använda teoretiska samband och eventuella härledningar.
  \item[Metod] \hfill \\ 
    Experimentell metod och analysförfarande.
  \item[Resultat och diskussion] \hfill \\ 
    De data ni erhållit från era analyser. Enskilda värden kan
    presenteras i flytande text, serier av värden förslagsvis i tabeller.
    Vid regressionsanalys kan en figur vara belysande. Ifall ni har
    ett stort antal figurer kan merparten av dem läggas i
    bilaga. Diskutera era resultat utifrån vad ni förväntade er och de
    approximationer ni gjort. I diskussionen finns visst utrymme för
    spekulationer.
  \item[Slutsatser] \hfill \\ 
    Sammanfatta det viktigaste av era resultat. Vad kan man med säkerhet
    säga (inga spekulationer).
  \item[Referenser] \hfill \\
    Referenser till data/teori ni själva inte bestämt/härlett.
  \item[Bilaga - kod för regressionsanalys] \hfill \\
    Kod för kurvanpassing (exempelvis Matlab skript).
  \item[Övriga bilagor] \hfill \\
    Exempelvis figurer för respektive mätserie med både rådata och
    passade kurvor. Passade parametrars värden i tabellform eller i
    etiketterna.
\end{description}

%%% Local Variables:
%%% mode: latex
%%% TeX-master: "../main"
%%% ispell-local-dictionary: "swedish"
%%% End:
