\section{Dataanalys}
\label{sec:analys}
Skriv ett skript (i exmepelvis Matlab, se exempelskript i
\cref{sec:matlab-exempel} i Bilaga) för att passa funktionsuttryck 
till era rådata. Gör passningen för både ``pseudo första ordningens''
och andra ordningens behandling i enlighet med härledningarna i
\cref{sec:irrev_unary,sec:irrev_binary}. Nedan följer några tips:

\begin{enumerate}
\item Kom ihåg att det är ett visst
  avstånd mellan blandkammaren i stopped-flow-utrustningen och
  spektrofotometern vilket betyder att $t_0$ är en parameter beroende på
  flödeshastigheten ni åstadkommer vid blandingen av reaktantlösningarna.
\item Hastighetsutrycket för pseudo första ordningens reaktion kan
  linjäriseras genom att göra passningen mot logaritmen av
  absorbansen. OLS\footnote{Ordinary Least Squares} passningen är
  då analytisk (``closed form'') vilket gör den lämplig att använda sig av för
  att bestämma initialgissning till den icke-linjära (iterativa)
  passningen som behövs för ``andra ordningens'' behandling.\footnote{
  Ni kan även göra en icke-linjär passning för det irreversibla fallet
  utgåendes från otransformerade data, dock kommer det troligtvis endast
  ha en marginell effekt på erhållna parametrar.}
\item Ni behöver inte göra separata passningar för det reversibla
  fallet. Förfarandet skiljer sig endast i den efterföljande
  ekvationslösningen (där jämviktskonstanten nu behövs för bestämning av
  $k_f$).
\end{enumerate}

%%% Local Variables:
%%% mode: latex
%%% TeX-master: "../main"
%%% ispell-local-dictionary: "swedish"
%%% End:
