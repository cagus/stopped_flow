\section{Instuderingsfrågor}
\label{sec:instudering}
Det är viktigt att ni innan laborationen har en idé om vad ni vill ta
reda på och hur ni vill gå till väga. Därför skall ni inför
laborationstillfället besvarat följande frågeställningar skriftligen:

\begin{enumerate}
\item Vad händer ifall ni försöker följa reaktionen vid för hög
  temperatur?
\item Vad händer ifall ni försöker följa reaktionen vid för hög
  jonstyrka?
\item Vad sker med jonstyrkan under reaktionsförloppet i en vattenlösning
  med stökiometriska mängder järn(III)perklorat och natriumtiocyanat med
  mycket små halter av åskådarjoner? Hur påverkar det
  hastighetskonstanten $k_f$?
\item Vad är tänkbara initialkoncentrationer (dvs. vad tror ni kommer att
  fungera bra)? Dessa kan sedan justeras baserat på erhållna data.
\item Vad är jonstyrkan vid $t=0$ och $t=\infty$ för era föreslagna
  koncentrationer?
\item Vad är absorbansen vid $t=\infty$ för era föreslagna
  koncentrationer? Vid vilken transmittans tror ni att ni har bäst
  känslighet i er detektor? Vilken absorbans motsvarar det?
\end{enumerate}

%%% Local Variables:
%%% mode: latex
%%% TeX-master: "../main"
%%% ispell-local-dictionary: "swedish"
%%% End:
