\section{Teori}
\label{sec:teori}
Laborationen introducerar ingen ny teori. Föreläsningarna samt
kurslitteratur för kursen innehåller det ni behöver. Men som ledning för
den numeriska analysen ges här fullständiga härledningar av
koncentrationernas tidsberoende för olika grad av förenkling av
hastighetsuttrycken.

\subsection{Hastighetsuttryck}
För läslighetens skull betecknar vi från och med nu [\ce{FeSCN^2+}],
[\ce{Fe^3+}] och [\ce{SCN-}] och med variablerna $x$, $y$ och $z$ och de
initiala koncentrationerna med respektive versaler $X$, $Y$ och $Z$. Vi
kan då beskriva koncentrationerna vid tiden $t$ enligt:

%\begin{table}
\begin{center}
\begin{tabular}{rccccc}
        & \ce{Fe^3+} & + & \ce{SCN-} & \ce{<=>>[k_f][k_b]} & \ce{FeSCN^2+} \\
  $t=0$ &     $Y$    &   &    $Z$    &                    &      $X$        \\
  $t>0$ &  $y=Y+X-x$ &   & $z=Z+X-x$ &                    &      $x$        \\
  $t>0, X=0$ &  $y=Y+x$ &   & $z=Z-x$ &                    &      $x$        \\
\end{tabular}
\end{center}
%\end{table}


härledningarna nedan baseras - för läslighets skull - på sista raden,
$X=0$, vilket motsvarar förhållandena i våra stopped flow försök (vi har
idealt ingen produkt innan mätning).

Ert mål är att genom kurvanpassning med efterföljande ekvationslösning
bestämma $k_f$ för ett antal temperaturer.

Man kan härleda ett explicit analytiskt uttryck för koncentrationen
av \ce{FeSCN^2+} genom att integrera den ordinära differentialekvation
som beskriver dess tidsberoende. Vi kan välja att beskriva denna med
olika grad av förenklingar.
\begin{description}
\item[Irreversibel] \hfill \\
  Om en jämvikt är kraftigt förskjuten mot produkt kommer
  bakåtreaktionen vara av liten betydelse och kan därmed
  ignoreras i behandlingen av kinetiken för att få
  ett enklare hastighetsuttryck.
\item[Pseduo första ordningen] \hfill \\ %
  Ifall initialkoncentrationen av en reaktant är mycket större än den andra
  kan reaktanten i överskott approximeras som konstant under reaktionens
  gång. Man bestämmer då istället den så kallade ``pseudo första ordningens''
  hastighets konstant $k' = k_f[\ce{Fe^3+}]$
\end{description}

\begin{table}
  \caption[Hastighetsuttryck för tiocyanatojärn(III)]{Hastighetsuttryck
    ($\frac{dx}{dt}$) med olika grad av förenkling. Antagandet $Y \gg Z $
  gäller för pseudo 1:a ordningens uttryck.} % [\ce{FeSCN^2+}]
  \label{tab:rate_eqs}
  \begin{center}
  \begin{tabular}{lll}
   \toprule
         {}
           &
         Irreversibel
           &
         Reversibel
       \tabularnewline
   \midrule
            Pseudo 1:a ordn.
               &
            $Y k_{f} \left(Z - x\right)$
               &
            $\input{derivations/rev_unary_rate}$
        \tabularnewline
             2:a ordn.
               &
            $k_{f} \left(Y - x\right) \left(Z - x\right)$
               &
            $- k_{b} x + k_{f} \left(Y - x\right) \left(Z - x\right)$
        \tabularnewline
   \bottomrule
  \end{tabular}
  \end{center}
  \footnotesize
\end{table}


I \cref{tab:rate_eqs} ges olika uttryck för $\frac{dx}{dt}$. I
\cref{sec:irrev_unary,sec:rev_unary,sec:irrev_binary,sec:rev_binary} 
följer härledningar för vart och ett av dessa fyra fall. 

% \begin{equation}
%   \label{eq:scn-rate}
%   \frac{d[\ce{SCN-}]}{dt} = k_b[\ce{FeSCN^2+}] - k_f[\ce{Fe^3+}][\ce{SCN-}]
% \end{equation}

% Frågan är nu hur man skall bestämma $k_f$. För det första behöver vi ett funktionsuttryck
% för [\ce{FeSCN^2+}]. Enklast är kanske i detta skede konstatera att under antagandet att
% vi inte har någon mängd \ce{FeSCN^2+} vid reaktionens start gäller:

% \begin{equation}
%   \label{eq:fescn-scn-rel}
%   [\ce{FeSCN^2+}] = [\ce{SCN-}]_0 - [\ce{SCN-}]
% \end{equation}

% vilket låter oss fokusera på att lösa \cref{eq:scn-rate}.

% Detta kan göras med olika grad av förenklingar. Låt oss studera dem nedan.

\subsubsection{Irreversibel pseudo första ordningens reaktion}
\label{sec:irrev_unary}
Detta är den enklaste modell vi kan ansätta (och även den med störst fel).
Jämviktskonstanten visar att den ``framåtgående'' reaktionen (bildandet av \ce{FeSCN^2+})
är den dominerande. Vidare har vi valt att låta $[\ce{Fe^3+}]_0=10[\ce{SCN-}]_0$ vilket
betyder att koncentrationen järn(III) är ganska konstant under reaktionens gång. Om vi
utnyttjar dessa antaganden får vi ett förenklat uttryck för utvecklingen
av \ce{SCN-}:

\begin{equation}
  \label{eq:scn-pseudo-rate}
  \frac{d[\ce{SCN-}]}{dt} = -k_f[\ce{Fe^3+}][\ce{SCN-}] = -k'[\ce{SCN-}]
\end{equation}

där $k' = k_f[\ce{Fe^3+}]$. \cref{eq:scn-pseudo-rate} har en enkel lösning:

\begin{align}
  \frac{d}{d t} x = Y k_{f} \left(Z - x\right) \\
  \int_{0}^{x} \frac{1}{Y k_{f} \left(Z - \chi\right)}\, d\chi = \int_{0}^{t} 1\, d\tau \\
  \frac{1}{Y k_{f}} \left(\log{\left (Z \right )} - \log{\left (Z - x \right )}\right) = t \\
  x = Z \left(1 - e^{- Y k_{f} t}\right)
\end{align}


\subsubsection{Reversibel pseudo första ordningens reaktion}
\label{sec:rev_unary}
Ett steg mot en mer korrekt beskrivning är att även beakta
bakåtreaktionen (vi betraktar fortfarande [\ce{Fe^3+}] som
konstant). Härledningen är analog och blir:
\begin{align}
  \frac{d}{d t} x = k'_{f} \left(Z - x\right) - k_{b} x \\
  \int_{0}^{x} \frac{1}{- \chi k_{b} + k'_{f} \left(Z - \chi\right)}\, d\chi = \int_{0}^{t} 1\, d\tau \\
  \frac{1}{k'_{f} + k_{b}} \left(\log{\left (- Z k'_{f} \right )} - \log{\left (- Z k'_{f} + x \left(k'_{f} + k_{b}\right) \right )}\right) = t \\
  x = \frac{Z k'_{f} \left(e^{t \left(k'_{f} + k_{b}\right)} - 1\right)}{\left(k'_{f} + k_{b}\right) e^{t \left(k'_{f} + k_{b}\right)}} \\
  x = \frac{Z k'_{f}}{k'_{f} + k_{b}} \left(1 - e^{- t \left(k'_{f} + k_{b}\right)}\right)
\end{align}



\subsubsection{Irreversibel bimolekylär kinetik}
\label{sec:irrev_binary}
Om vi istället fokuserar på att bättre beskriva effekten av att
[\ce{Fe^3+}] är tidsberoende får vi en något svårare integral att lösa:

\begin{align}
  \frac{d}{d t} x = k_{f} \left(Y - x\right) \left(Z - x\right) \\
  \int_{0}^{x} \frac{1}{\left(Y - \chi\right) \left(Z - \chi\right)}\, d\chi = \int_{0}^{t} k_{f}\, d\tau
\end{align}

Den primitiva funktionen kan vi erhålla genom att slå upp den i ``BETA Handbook of
Mathematics'', integrera för hand med partialbråks 
uppdelning eller använda ett CAS\footnote{  Computer Algebra System -
  exempelvis Mathematica, Maple, Maxima, SymPy m.fl.}. Friställandet av
vår beroende variabel $x$ till ett explicit uttryck i den oberoende
variabeln $t$ blir:  

\begin{align}
  \frac{1}{Y - Z} \left(\log{\left (\frac{Z}{Y} \right )} + \log{\left (Y - x \right )} - \log{\left (Z - x \right )}\right) = k_{f} t \\
  x = \frac{Y \left(- e^{k_{f} t \left(- Y + Z\right)} + 1\right)}{\frac{Y}{Z} - e^{k_{f} t \left(- Y + Z\right)}}
\end{align}


vi ser att det detta explicita uttrycket inte längre kan linjäriseras
vilket gör en regression mer avancerad.

\subsubsection{Reversibel bimolekylär kinetik}
Slutligen härleder vi den mest korrekta behandlingen av vårt kinetikproblem.
\label{sec:rev_binary}
\begin{align}
  \frac{d}{d t} x = - k_{b} x + k_{f} \left(Y - x\right) \left(Z - x\right) \\
  \int_{0}^{x} \frac{1}{- \chi k_{b} + k_{f} \left(Y - \chi\right) \left(Z - \chi\right)}\, d\chi = \int_{0}^{t} 1\, d\tau
\end{align}
\begin{align}
  \int \frac{1}{a x^{2} + b x + c}\, dx = \begin{cases} C + \frac{1}{\sqrt{- 4 a c + b^{2}}} \log{\left (\frac{2 a x + b - \sqrt{- 4 a c + b^{2}}}{2 a x + b + \sqrt{- 4 a c + b^{2}}} \right )} & \text{for}\: 4 a c < b^{2} \\C - \frac{2}{2 a x + b} & \text{for}\: 4 a c = b^{2} \\C + \frac{2}{\sqrt{4 a c - b^{2}}} \operatorname{atan}{\left (\frac{2 a x + b}{\sqrt{4 a c - b^{2}}} \right )} & \text{for}\: 4 a c > b^{2} \end{cases}
\end{align}

\begin{align}
  \begin{Bmatrix}a : k_{f}, & b : - Y k_{f} - Z k_{f} - k_{b}, & c : Y Z k_{f}\end{Bmatrix} \\
  P = \sqrt{- 4 a c + b^{2}} \\
  \frac{1}{P} \left(- \log{\left (\frac{- P + b}{P + b} \right )} + \log{\left (\frac{- P + 2 a x + b}{P + 2 a x + b} \right )}\right) = t \\
  x = \frac{\left(P - b\right) \left(- P e^{P t} + P - b e^{P t} + b\right)}{2 a \left(P + b + \left(P - b\right) e^{P t}\right)} \\
  Q = P + b \\
  R = P - b \\
  x = - \frac{Q \left(e^{P t} - 1\right)}{2 a \left(\frac{Q}{R} + e^{P t}\right)}
\end{align}



\subsection{Hastighetskontanter}
Hastighetskonstanter är verkliga konstanter för en given temperatur och
jonstyrka. I era försök kommer dessa parametrar att variera och det är
upp till er att behandla dessa effekter enligt de modeller som
behandlas i kursen.

\subsection{Förenklingar}
I denna laboration antar beaktar vi endast reaktionen för bildning av
monothiocyanatojärn(II). Men i verkligheten förekommer även följande
komplex: 

\begin{center}
  \begin{tabular}{ll}
    \ce{FeSCN^2+ + SCN- <=> Fe(SCN)_2^+}  & K=\SI{4.9}{\per\Molar} \\
    \ce{Fe(SCN)_2^+ + SCN- <=> Fe(SCN)3}  & K=\SI{5.1}{\per\Molar} \\
    \ce{Fe^3+ + H2O <=> FeOH^2+ + H+}     & $\log K = \num{-2.19}$ \\
    \ce{Fe^3+ + 2H2O <=> Fe2OH4^2- + 2H+} & $\log K = \num{-2.95}$ \\
  \end{tabular}
\end{center}

Hur stora dessa effekter är beror som man ser på [\ce{SCN-}] och pH. Ni
bör därför ta detta i beaktande när ni väljer era koncentrationer.

%%% Local Variables:
%%% mode: latex
%%% TeX-master: "../main"
%%% End:
