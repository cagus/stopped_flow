\section{Inledning}
\label{sec:inledning}
I denna laboration skall ni, i vattenlösning, studera hastigheten för
(framåt-) reaktionen mellan järn(III)joner och tiocyanatjoner:

\begin{center}
\begin{tabular}{ccc}
  \ce{Fe^3+ + SCN- <=>>[k_f][k_b] FeSCN^2+} & %
    $\log_{10} \left( \beta_1 %
     % (I=\SIrange{1.0}{1.2}{\molar}) %
    / \si{\per\Molar} \right) = \num{2.065(5)}$ %
  % & $\Delta H_1 = \SI{-6.7}{\kilo\joule\per\mole}$
\end{tabular}
\end{center}
Komplexbildningskonstanten för tiocyanatojärn(III) är från
Referens\cite{peintler_improved_2000}\mbox{.}

\ce{FeSCN^2+} är starkt rödfärgat ($\lambda_{max}=\SI{480}{\nm};
~\varepsilon_{\SI{480}{\nm}} =
\SI{5148}{\per\Molar\per\centi\metre}$)
\cite{peintler_improved_2000} medan reaktanterna är (relativt) färglösa. Det finns
komplex med fler än en tiocyanatjon, men så länge koncentrationen av
\ce{SCN-} är tillräckligt låg kan vi förbise dessa.

\subsection{Mål}
Målet med laborationen är att ni självständigt skall undersöka både
temperatur- och jonstyrkeberoendet av hastighetskonstanten för
komplexbildningsreaktionen. Slutligen skall ni redovisa era resultat och
slutsatser i en laborationsrapport som följer strukturen för en
vetenskaplig artikel.

\subsection{Förberedelser}
Hur ni väljer att analysera temperatur- och jonstyrkeberoende är
upp till er, men förslagsvis utgår ni från väletablerade
teorier som introducerats i denna kurs. Utöver val av teoretisk behandling
kommer ni själva få välja experimentella förhållanden vid era försök.

Detta betyder att denna laboration är av en mer utmanande karaktär och
kräver troligen att ni avsätter avsevärd tid innan laborationen för att
beräkna lämpliga intervall för koncentrationer, jonstyrka och
temperatur. Den teoretiska behandling ställer vissa krav på kvalitet (och
kvantitet) av insamlade data. Det är därför viktigt att ni har en plan
för hur ni skall optimera parametrarna under genomförandet av laborationen.

Innan laborationstillfället skall ni också ha gjort de
instuderingsuppgifter som finns i \cref{sec:instudering}. Tag med
skriftliga svar på dessa för inlämning till laboratorieassistenten vid
laborationstillfället.

%%% Local Variables:
%%% mode: latex
%%% TeX-master: "../main"
%%% ispell-local-dictionary: "swedish"
%%% End:
